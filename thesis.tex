\documentclass[12pt,a4paper]{article}
\usepackage[english, ukrainian]{babel}
\usepackage[utf8]{inputenc}
\usepackage[T2A]{fontenc}
\usepackage[left=3cm,top=3cm,right=1.5cm,bottom=1.5cm,nohead,includefoot]{geometry}
\usepackage{setspace}
\usepackage{listings}
\usepackage{color}
\usepackage{float}
\usepackage[pdftex]{graphicx}
\usepackage{courier}
\usepackage{bold-extra}
\usepackage{fix-cm}
\usepackage{alltt}
\usepackage{indentfirst}
\usepackage{amsmath, amsthm, amssymb}
\usepackage{url}

\setstretch{1.1}

\begin{document}
\pretolerance=-1
\tolerance=2300

\thispagestyle{empty}
\setlength{\parindent}{1.5cm}
\fontsize{14pt}{6mm}\selectfont

\begin{center}
  Міністерство освіти і науки, молоді та спорту України
  
  Львівський національний університет імені Івана Франка

  Факультет прикладної математики та інформатики

  Кафедра програмування\\[5cm]

  {\bfseries\Large Магістерська робота}\\[0.5cm]
  на тему:\\[0.5cm]
  {\bfseries\Large Розширення функціональності моделі FAMIX для побудови абстрактних дерев коду Java- та Smalltalk-програм}\\[5cm]

  \begin{flushleft}\leftskip11cm
    Виконав:\\
    студент групи ПМІ-51м\\
    Тимчук Юрій\linebreak

    Науковий керівник:\\
    доц. Рикалюк Р.Є.
    
  \end{flushleft}

  \vspace{2cm}
  Львів - 2013 
\end{center}

\clearpage

\setstretch{1.5}
\fontsize{14pt}{6mm}\selectfont

\newcommand{\vect}[1]{(#1_1,#1_2,\dots,#1_n)}

\pagenumbering{roman}
\tableofcontents
\clearpage
\pagenumbering{arabic}

\section{Вступ}

Метамоделювання було створено, щоб дозволити розглядати програми на більш високому рівні абстракції, ніж це дозволяють поточні мови програмування. Основна ідея полягає в тому, що хтось визначає високорівневу модель рішення, описує, як перетворити це рішення в програму для даної мови, а потім автоматично генерує вихідний код. В ідеалі такий же підхід дозволяє взяти існуючу програму, автоматично отримати абстрактну модель з неї, вручну розширити або поліпшити модель і згенерувати нову програму, можливо, на новій мові програмування (round-trip engineering). Такий підхід був би дуже цікавим для організацій, що мають програми написані на старих технологіях (наприклад, Cobol) і хочуть перенести їх на більш сучасні (наприклад, Java). Команда RMod вже має інструменти, які можуть приймати програми (написані на C, Java, Smalltalk, і т.д.) в якості вхідних даних і продукувати з них метамоделі.

Тим не менш, чорт сидить в деталях, в даному випадку, для зворотньої розробки існуючої програми, щоб мати можливість відновити її потім, треба тримати багато деталей по цю програму. Ця потреба, звичайно, не сумісна з ідеєю абстрагування моделі цієї програми. Тому потрібно, бути в змозі створити абстрактну модель програми, але, в той же час, створити дуже детальну модель тієї ж програми і таким чином, мати можливість працювати на двох рівнях абстракції одночасно.

Як приклад можна розглянути дві об'єктно-орієнтовані мови програмування: Java та Smalltalk. Обидві мають подібну структуру: простори імен, класи, поля та методи класів. Але у мові Smalltalk відсутні інструкції галуження та циклу, які, безперечно, відіграють важливу роль в Java коді. З другого боку код Java не має нічого аналогічного до блоків у мові Smalltalk, які дозволяють реалізовувати галуження та цикли за допомогою методів.

Метою цієї магістерської роботи є розширення вже існуючої метамоделі FAMIX шляхом додавання до неї компонент абстрактних синтаксичних дерев мов програмування Java та Smalltalk.

\clearpage
\addcontentsline{toc}{section}{Література}
\begin{thebibliography}{9}


\end{thebibliography}

\end{document}
