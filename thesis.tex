\documentclass[12pt,a4paper]{article}

\usepackage{fontspec}
\usepackage{polyglossia}
\usepackage[left=2.5cm,top=2.5cm,right=2.5cm,bottom=2.5cm,nohead]{geometry}
\usepackage{setspace}
\usepackage{listings}
\usepackage{color}
\usepackage{float}
\usepackage{courier}
\usepackage{bold-extra}
\usepackage{fix-cm}
\usepackage{alltt}
\usepackage{indentfirst}
\usepackage{amsmath, amsthm, amssymb}
\usepackage{url}

\defaultfontfeatures{Mapping=tex-text}

\setmainfont
    [ Path           = fonts/ ,
      UprightFont    = *-Regular,
      BoldFont       = *-Bold ,
      ItalicFont     = *-Italic ,
      BoldItalicFont = *-BoldItalic]
    {LiberationSerif}
\setsansfont
    [ Path           = fonts/ ,
      UprightFont    = *-Regular,
      BoldFont       = *-Bold ,
      ItalicFont     = *-Italic ,
      BoldItalicFont = *-BoldItalic]
    {LiberationSans}
\setmonofont
    [ Path           = fonts/ ,
      UprightFont    = *-Regular,
      BoldFont       = *-Bold ,
      ItalicFont     = *-Italic ,
      BoldItalicFont = *-BoldItalic]
    {LiberationMono}

\setmainlanguage{ukrainian}
\setotherlanguage{english}

\setstretch{1.1}

\begin{document}
\pretolerance=-1
\tolerance=2300

\pagenumbering{arabic}
\pagestyle{empty}
\setlength{\parindent}{1.5cm}
\fontsize{14pt}{6mm}\selectfont

\begin{center}
  Міністерство освіти і науки, молоді та спорту України
  
  Львівський національний університет імені Івана Франка

  Факультет прикладної математики та інформатики
\end{center}

\vspace{1cm}

\begin{flushright}
  Кафедра програмування
\end{flushright}

\vspace{4cm}

\begin{center}
  {\bfseries\Large <ТЕМА>}
\end{center}

\vspace{2cm}

\begin{small}
\begin{flushleft}\leftskip8.5cm
  <НАУКОВА> робота студента(ки) групи <ГРУПА>\\
  <ПРИЗВІЩЕ ІНІЦІАЛИ (р. відм.)>\linebreak
  
  Науковий керівник:\\
  <СТУПІНЬ>\\
  <ПРИЗВІЩЕ ІНІЦІАЛИ>
\end{flushleft}
\end{small}

\vspace{4cm}

\begin{center}
  Львів - <РІК> 
\end{center}

\clearpage



\setstretch{1.5}
\fontsize{14pt}{6mm}\selectfont

\tableofcontents
\clearpage
\pagestyle{plain}
\section{Вступ}

Лорем іпсум!\cite{alias}

\clearpage

\section{<РОЗДІЛ \#>}

І так далі\cite{web}

\clearpage
\addcontentsline{toc}{section}{Література}
\begin{thebibliography}{9}

  \bibitem{alias}<АВТОР> \emph{<КНИГА>},
    <ВИДАВНИЦТВО> <РІК>, <К-КІСТЬ СТОРІНОК> ст.
    
  \bibitem{web}<АВТОР> \emph{<НАЗВА>} [Електронний ресурс],
    <РІК>. Режим доступу:
    \url{https://github.com/Uko/thesis-template}

\end{thebibliography}

\end{document}
